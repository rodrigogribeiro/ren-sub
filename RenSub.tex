\documentclass{article}
\usepackage{a4}
\usepackage{palatino}
\usepackage{natbib}
\usepackage{amsfonts}
\usepackage{stmaryrd}
\usepackage{upgreek}
\usepackage{url}


\DeclareMathAlphabet{\mathkw}{OT1}{cmss}{bx}{n}

\usepackage{color}
\newcommand{\redFG}[1]{\textcolor[rgb]{0.6,0,0}{#1}}
\newcommand{\greenFG}[1]{\textcolor[rgb]{0,0.4,0}{#1}}
\newcommand{\blueFG}[1]{\textcolor[rgb]{0,0,0.8}{#1}}
\newcommand{\orangeFG}[1]{\textcolor[rgb]{0.8,0.4,0}{#1}}
\newcommand{\purpleFG}[1]{\textcolor[rgb]{0.4,0,0.4}{#1}}
\newcommand{\yellowFG}[1]{\textcolor{yellow}{#1}}
\newcommand{\brownFG}[1]{\textcolor[rgb]{0.5,0.2,0.2}{#1}}
\newcommand{\blackFG}[1]{\textcolor[rgb]{0,0,0}{#1}}
\newcommand{\whiteFG}[1]{\textcolor[rgb]{1,1,1}{#1}}
\newcommand{\yellowBG}[1]{\colorbox[rgb]{1,1,0.2}{#1}}
\newcommand{\brownBG}[1]{\colorbox[rgb]{1.0,0.7,0.4}{#1}}

\newcommand{\ColourStuff}{
  \newcommand{\red}{\redFG}
  \newcommand{\green}{\greenFG}
  \newcommand{\blue}{\blueFG}
  \newcommand{\orange}{\orangeFG}
  \newcommand{\purple}{\purpleFG}
  \newcommand{\yellow}{\yellowFG}
  \newcommand{\brown}{\brownFG}
  \newcommand{\black}{\blackFG}
  \newcommand{\white}{\whiteFG}
}

\newcommand{\MonochromeStuff}{
  \newcommand{\red}{\blackFG}
  \newcommand{\green}{\blackFG}
  \newcommand{\blue}{\blackFG}
  \newcommand{\orange}{\blackFG}
  \newcommand{\purple}{\blackFG}
  \newcommand{\yellow}{\blackFG}
  \newcommand{\brown}{\blackFG}
  \newcommand{\black}{\blackFG}
  \newcommand{\white}{\blackFG}
}

\ColourStuff


\newcommand{\M}[1]{\mathsf{#1}}
\newcommand{\D}[1]{\blue{\mathsf{#1}}}
\newcommand{\C}[1]{\red{\mathsf{#1}}}
\newcommand{\F}[1]{\green{\mathsf{#1}}}
\newcommand{\V}[1]{\purple{\mathit{#1}}}
\newcommand{\T}[1]{\raisebox{0.02in}{\tiny\green{\textsc{#1}}}}

\newcommand{\us}[1]{\_\!#1\!\_}

%% ODER: format ==         = "\mathrel{==}"
%% ODER: format /=         = "\neq "
%
%
\makeatletter
\@ifundefined{lhs2tex.lhs2tex.sty.read}%
  {\@namedef{lhs2tex.lhs2tex.sty.read}{}%
   \newcommand\SkipToFmtEnd{}%
   \newcommand\EndFmtInput{}%
   \long\def\SkipToFmtEnd#1\EndFmtInput{}%
  }\SkipToFmtEnd

\newcommand\ReadOnlyOnce[1]{\@ifundefined{#1}{\@namedef{#1}{}}\SkipToFmtEnd}
\usepackage{amstext}
\usepackage{amssymb}
\usepackage{stmaryrd}
\DeclareFontFamily{OT1}{cmtex}{}
\DeclareFontShape{OT1}{cmtex}{m}{n}
  {<5><6><7><8>cmtex8
   <9>cmtex9
   <10><10.95><12><14.4><17.28><20.74><24.88>cmtex10}{}
\DeclareFontShape{OT1}{cmtex}{m}{it}
  {<-> ssub * cmtt/m/it}{}
\newcommand{\texfamily}{\fontfamily{cmtex}\selectfont}
\DeclareFontShape{OT1}{cmtt}{bx}{n}
  {<5><6><7><8>cmtt8
   <9>cmbtt9
   <10><10.95><12><14.4><17.28><20.74><24.88>cmbtt10}{}
\DeclareFontShape{OT1}{cmtex}{bx}{n}
  {<-> ssub * cmtt/bx/n}{}
\newcommand{\tex}[1]{\text{\texfamily#1}}	% NEU

\newcommand{\Sp}{\hskip.33334em\relax}


\newcommand{\Conid}[1]{\mathit{#1}}
\newcommand{\Varid}[1]{\mathit{#1}}
\newcommand{\anonymous}{\kern0.06em \vbox{\hrule\@width.5em}}
\newcommand{\plus}{\mathbin{+\!\!\!+}}
\newcommand{\bind}{\mathbin{>\!\!\!>\mkern-6.7mu=}}
\newcommand{\rbind}{\mathbin{=\mkern-6.7mu<\!\!\!<}}% suggested by Neil Mitchell
\newcommand{\sequ}{\mathbin{>\!\!\!>}}
\renewcommand{\leq}{\leqslant}
\renewcommand{\geq}{\geqslant}
\usepackage{polytable}

%mathindent has to be defined
\@ifundefined{mathindent}%
  {\newdimen\mathindent\mathindent\leftmargini}%
  {}%

\def\resethooks{%
  \global\let\SaveRestoreHook\empty
  \global\let\ColumnHook\empty}
\newcommand*{\savecolumns}[1][default]%
  {\g@addto@macro\SaveRestoreHook{\savecolumns[#1]}}
\newcommand*{\restorecolumns}[1][default]%
  {\g@addto@macro\SaveRestoreHook{\restorecolumns[#1]}}
\newcommand*{\aligncolumn}[2]%
  {\g@addto@macro\ColumnHook{\column{#1}{#2}}}

\resethooks

\newcommand{\onelinecommentchars}{\quad-{}- }
\newcommand{\commentbeginchars}{\enskip\{-}
\newcommand{\commentendchars}{-\}\enskip}

\newcommand{\visiblecomments}{%
  \let\onelinecomment=\onelinecommentchars
  \let\commentbegin=\commentbeginchars
  \let\commentend=\commentendchars}

\newcommand{\invisiblecomments}{%
  \let\onelinecomment=\empty
  \let\commentbegin=\empty
  \let\commentend=\empty}

\visiblecomments

\newlength{\blanklineskip}
\setlength{\blanklineskip}{0.66084ex}

\newcommand{\hsindent}[1]{\quad}% default is fixed indentation
\let\hspre\empty
\let\hspost\empty
\newcommand{\NB}{\textbf{NB}}
\newcommand{\Todo}[1]{$\langle$\textbf{To do:}~#1$\rangle$}

\EndFmtInput
\makeatother
%
%
%
%
%
%
% This package provides two environments suitable to take the place
% of hscode, called "plainhscode" and "arrayhscode". 
%
% The plain environment surrounds each code block by vertical space,
% and it uses \abovedisplayskip and \belowdisplayskip to get spacing
% similar to formulas. Note that if these dimensions are changed,
% the spacing around displayed math formulas changes as well.
% All code is indented using \leftskip.
%
% Changed 19.08.2004 to reflect changes in colorcode. Should work with
% CodeGroup.sty.
%
\ReadOnlyOnce{polycode.fmt}%
\makeatletter

\newcommand{\hsnewpar}[1]%
  {{\parskip=0pt\parindent=0pt\par\vskip #1\noindent}}

% can be used, for instance, to redefine the code size, by setting the
% command to \small or something alike
\newcommand{\hscodestyle}{}

% The command \sethscode can be used to switch the code formatting
% behaviour by mapping the hscode environment in the subst directive
% to a new LaTeX environment.

\newcommand{\sethscode}[1]%
  {\expandafter\let\expandafter\hscode\csname #1\endcsname
   \expandafter\let\expandafter\endhscode\csname end#1\endcsname}

% "compatibility" mode restores the non-polycode.fmt layout.

\newenvironment{compathscode}%
  {\par\noindent
   \advance\leftskip\mathindent
   \hscodestyle
   \let\\=\@normalcr
   \let\hspre\(\let\hspost\)%
   \pboxed}%
  {\endpboxed\)%
   \par\noindent
   \ignorespacesafterend}

\newcommand{\compaths}{\sethscode{compathscode}}

% "plain" mode is the proposed default.
% It should now work with \centering.
% This required some changes. The old version
% is still available for reference as oldplainhscode.

\newenvironment{plainhscode}%
  {\hsnewpar\abovedisplayskip
   \advance\leftskip\mathindent
   \hscodestyle
   \let\hspre\(\let\hspost\)%
   \pboxed}%
  {\endpboxed%
   \hsnewpar\belowdisplayskip
   \ignorespacesafterend}

\newenvironment{oldplainhscode}%
  {\hsnewpar\abovedisplayskip
   \advance\leftskip\mathindent
   \hscodestyle
   \let\\=\@normalcr
   \(\pboxed}%
  {\endpboxed\)%
   \hsnewpar\belowdisplayskip
   \ignorespacesafterend}

% Here, we make plainhscode the default environment.

\newcommand{\plainhs}{\sethscode{plainhscode}}
\newcommand{\oldplainhs}{\sethscode{oldplainhscode}}
\plainhs

% The arrayhscode is like plain, but makes use of polytable's
% parray environment which disallows page breaks in code blocks.

\newenvironment{arrayhscode}%
  {\hsnewpar\abovedisplayskip
   \advance\leftskip\mathindent
   \hscodestyle
   \let\\=\@normalcr
   \(\parray}%
  {\endparray\)%
   \hsnewpar\belowdisplayskip
   \ignorespacesafterend}

\newcommand{\arrayhs}{\sethscode{arrayhscode}}

% The mathhscode environment also makes use of polytable's parray 
% environment. It is supposed to be used only inside math mode 
% (I used it to typeset the type rules in my thesis).

\newenvironment{mathhscode}%
  {\parray}{\endparray}

\newcommand{\mathhs}{\sethscode{mathhscode}}

% texths is similar to mathhs, but works in text mode.

\newenvironment{texthscode}%
  {\(\parray}{\endparray\)}

\newcommand{\texths}{\sethscode{texthscode}}

% The framed environment places code in a framed box.

\def\codeframewidth{\arrayrulewidth}
\RequirePackage{calc}

\newenvironment{framedhscode}%
  {\parskip=\abovedisplayskip\par\noindent
   \hscodestyle
   \arrayrulewidth=\codeframewidth
   \tabular{@{}|p{\linewidth-2\arraycolsep-2\arrayrulewidth-2pt}|@{}}%
   \hline\framedhslinecorrect\\{-1.5ex}%
   \let\endoflinesave=\\
   \let\\=\@normalcr
   \(\pboxed}%
  {\endpboxed\)%
   \framedhslinecorrect\endoflinesave{.5ex}\hline
   \endtabular
   \parskip=\belowdisplayskip\par\noindent
   \ignorespacesafterend}

\newcommand{\framedhslinecorrect}[2]%
  {#1[#2]}

\newcommand{\framedhs}{\sethscode{framedhscode}}

% The inlinehscode environment is an experimental environment
% that can be used to typeset displayed code inline.

\newenvironment{inlinehscode}%
  {\(\def\column##1##2{}%
   \let\>\undefined\let\<\undefined\let\\\undefined
   \newcommand\>[1][]{}\newcommand\<[1][]{}\newcommand\\[1][]{}%
   \def\fromto##1##2##3{##3}%
   \def\nextline{}}{\) }%

\newcommand{\inlinehs}{\sethscode{inlinehscode}}

% The joincode environment is a separate environment that
% can be used to surround and thereby connect multiple code
% blocks.

\newenvironment{joincode}%
  {\let\orighscode=\hscode
   \let\origendhscode=\endhscode
   \def\endhscode{\def\hscode{\endgroup\def\@currenvir{hscode}\\}\begingroup}
   %\let\SaveRestoreHook=\empty
   %\let\ColumnHook=\empty
   %\let\resethooks=\empty
   \orighscode\def\hscode{\endgroup\def\@currenvir{hscode}}}%
  {\origendhscode
   \global\let\hscode=\orighscode
   \global\let\endhscode=\origendhscode}%

\makeatother
\EndFmtInput
%



\newcommand{\nudge}[1]{\marginpar{\footnotesize #1}}



\parskip 0.1in
\parindent 0in

\begin{document}
   \title{Type safe renaming and substitution for simple typed $\lambda$-calculus}

   \author{Rodrigo Ribeiro}

   \maketitle

I resolved to implement this in Agda as a way to:
\begin{itemize}
  \item Learn how to use lhs2TeX with literate Agda code a la McBride.
  \item Study this draft of McBride in order to develop things in Agda latter!
\end{itemize}

The first point here is the definition of types. We will consider types just
formed by some base type and arrow's:

   
\begin{hscode}\SaveRestoreHook
\column{B}{@{}>{\hspre}l<{\hspost}@{}}%
\column{3}{@{}>{\hspre}l<{\hspost}@{}}%
\column{9}{@{}>{\hspre}l<{\hspost}@{}}%
\column{E}{@{}>{\hspre}l<{\hspost}@{}}%
\>[B]{}\mathkw{data}\;\D{Ty}\;\mathbin{:}\;\D{Set}\;\mathkw{where}{}\<[E]%
\\
\>[B]{}\hsindent{3}{}\<[3]%
\>[3]{}\C{\upiota}\;{}\<[9]%
\>[9]{}\mathbin{:}\;\D{Ty}{}\<[E]%
\\
\>[B]{}\hsindent{3}{}\<[3]%
\>[3]{}\_\!\C{\rhd}\!\_\;\mathbin{:}\;\D{Ty}\;\blue{\rightarrow}\;\D{Ty}\;\blue{\rightarrow}\;\D{Ty}{}\<[E]%
\\[\blanklineskip]%
\>[B]{}\mathkw{infixr}\;\V{5}\;\_\!\C{\rhd}\!\_{}\<[E]%
\ColumnHook
\end{hscode}\resethooks

Next, we need contexts to hold type information for terms in DeBruijn notation.
Contexts are just snoc-lists of types.

\begin{hscode}\SaveRestoreHook
\column{B}{@{}>{\hspre}l<{\hspost}@{}}%
\column{3}{@{}>{\hspre}l<{\hspost}@{}}%
\column{8}{@{}>{\hspre}l<{\hspost}@{}}%
\column{E}{@{}>{\hspre}l<{\hspost}@{}}%
\>[B]{}\mathkw{data}\;\D{Cx}\;\mathbin{:}\;\D{Set}\;\mathkw{where}{}\<[E]%
\\
\>[B]{}\hsindent{3}{}\<[3]%
\>[3]{}\C{\mathcal{E}}\;{}\<[8]%
\>[8]{}\mathbin{:}\;\D{Cx}{}\<[E]%
\\
\>[B]{}\hsindent{3}{}\<[3]%
\>[3]{}\anonymous \!\raisebox{ -0.09in}[0in][0in]{\red{\textsf{`}}\,}\anonymous \;\mathbin{:}\;\D{Cx}\;\blue{\rightarrow}\;\D{Ty}\;\blue{\rightarrow}\;\D{Cx}{}\<[E]%
\\[\blanklineskip]%
\>[B]{}\mathkw{infixl}\;\V{4}\;\anonymous \!\raisebox{ -0.09in}[0in][0in]{\red{\textsf{`}}\,}\anonymous {}\<[E]%
\ColumnHook
\end{hscode}\resethooks

Now, DeBruijn indexes are just context membership evidence!

\begin{hscode}\SaveRestoreHook
\column{B}{@{}>{\hspre}l<{\hspost}@{}}%
\column{3}{@{}>{\hspre}l<{\hspost}@{}}%
\column{9}{@{}>{\hspre}l<{\hspost}@{}}%
\column{41}{@{}>{\hspre}l<{\hspost}@{}}%
\column{E}{@{}>{\hspre}l<{\hspost}@{}}%
\>[B]{}\mathkw{data}\;\us{\D{\in}}\;(\V{\tau}\;\mathbin{:}\;\D{Ty})\;\mathbin{:}\;\D{Cx}\;\blue{\rightarrow}\;\D{Set}\;\mathkw{where}{}\<[E]%
\\
\>[B]{}\hsindent{3}{}\<[3]%
\>[3]{}\V{here}\;{}\<[9]%
\>[9]{}\mathbin{:}\;\mathkw{forall}\;\{\mskip1.5mu \V{\Gamma}\mskip1.5mu\}\;{}\<[41]%
\>[41]{}\blue{\rightarrow}\;\V{\tau}\;\D{\in}\;\V{\Gamma}\;\!\raisebox{ -0.09in}[0in][0in]{\red{\textsf{`}}\,}\;\V{\tau}{}\<[E]%
\\
\>[B]{}\hsindent{3}{}\<[3]%
\>[3]{}\V{there}\;\mathbin{:}\;\mathkw{forall}\;\{\mskip1.5mu \V{\Gamma}\;\V{\sigma}\mskip1.5mu\}\;\blue{\rightarrow}\;\V{\tau}\;\D{\in}\;\V{\Gamma}\;\blue{\rightarrow}\;\V{\tau}\;\D{\in}\;\V{\Gamma}\;\!\raisebox{ -0.09in}[0in][0in]{\red{\textsf{`}}\,}\;\V{\sigma}{}\<[E]%
\\[\blanklineskip]%
\>[B]{}\mathkw{infix}\;\V{3}\;\us{\D{\in}}{}\<[E]%
\ColumnHook
\end{hscode}\resethooks

That done, we can build well typed terms by writing syntax-directed
rules for the typing judgment.

\newcommand{\negs}{\hspace*{ -0.3in}}
\begin{hscode}\SaveRestoreHook
\column{B}{@{}>{\hspre}l<{\hspost}@{}}%
\column{3}{@{}>{\hspre}l<{\hspost}@{}}%
\column{10}{@{}>{\hspre}l<{\hspost}@{}}%
\column{14}{@{}>{\hspre}l<{\hspost}@{}}%
\column{33}{@{}>{\hspre}l<{\hspost}@{}}%
\column{37}{@{}>{\hspre}l<{\hspost}@{}}%
\column{E}{@{}>{\hspre}l<{\hspost}@{}}%
\>[B]{}\mathkw{data}\;\us{\D{\vdash}}\;(\V{\Gamma}\;\mathbin{:}\;\D{Cx})\;\mathbin{:}\;\D{Ty}\;\blue{\rightarrow}\;\D{Set}\;\mathkw{where}{}\<[E]%
\\[\blanklineskip]%
\>[B]{}\hsindent{3}{}\<[3]%
\>[3]{}\C{var}\;\mathbin{:}\;{}\<[10]%
\>[10]{}\mathkw{forall}\;\{\mskip1.5mu \V{\tau}\mskip1.5mu\}{}\<[E]%
\\
\>[10]{}\blue{\rightarrow}\;{}\<[14]%
\>[14]{}\V{\tau}\;\D{\in}\;\V{\Gamma}{}\<[E]%
\\
\>[10]{}\mbox{\onelinecomment   \negs -------------}{}\<[E]%
\\
\>[10]{}\blue{\rightarrow}\;{}\<[14]%
\>[14]{}\V{\Gamma}\;\D{\vdash}\;\V{\tau}{}\<[E]%
\\[\blanklineskip]%
\>[B]{}\hsindent{3}{}\<[3]%
\>[3]{}\C{lam}\;\mathbin{:}\;{}\<[10]%
\>[10]{}\mathkw{forall}\;\{\mskip1.5mu \V{\sigma}\;\V{\tau}\mskip1.5mu\}{}\<[E]%
\\
\>[10]{}\blue{\rightarrow}\;{}\<[14]%
\>[14]{}\V{\Gamma}\;\!\raisebox{ -0.09in}[0in][0in]{\red{\textsf{`}}\,}\;\V{\sigma}\;\D{\vdash}\;\V{\tau}{}\<[E]%
\\
\>[10]{}\mbox{\onelinecomment   \negs ------------------}{}\<[E]%
\\
\>[10]{}\blue{\rightarrow}\;{}\<[14]%
\>[14]{}\V{\Gamma}\;\D{\vdash}\;\V{\sigma}\;\C{\rhd}\;\V{\tau}{}\<[E]%
\\[\blanklineskip]%
\>[B]{}\hsindent{3}{}\<[3]%
\>[3]{}\C{app}\;\mathbin{:}\;{}\<[10]%
\>[10]{}\mathkw{forall}\;\{\mskip1.5mu \V{\sigma}\;\V{\tau}\mskip1.5mu\}{}\<[E]%
\\
\>[10]{}\blue{\rightarrow}\;{}\<[14]%
\>[14]{}\V{\Gamma}\;\D{\vdash}\;\V{\sigma}\;\C{\rhd}\;\V{\tau}\;{}\<[33]%
\>[33]{}\blue{\rightarrow}\;{}\<[37]%
\>[37]{}\V{\Gamma}\;\D{\vdash}\;\V{\sigma}{}\<[E]%
\\
\>[10]{}\mbox{\onelinecomment   \negs -------------------------------}{}\<[E]%
\\
\>[10]{}\blue{\rightarrow}\;{}\<[14]%
\>[14]{}\V{\Gamma}\;\D{\vdash}\;\V{\tau}{}\<[E]%
\\[\blanklineskip]%
\>[B]{}\mathkw{infix}\;\V{3}\;\us{\D{\vdash}}{}\<[E]%
\ColumnHook
\end{hscode}\resethooks

Now, I can start the real business here: definition of substitution and renaming operations.

\begin{hscode}\SaveRestoreHook
\column{B}{@{}>{\hspre}l<{\hspost}@{}}%
\column{14}{@{}>{\hspre}l<{\hspost}@{}}%
\column{E}{@{}>{\hspre}l<{\hspost}@{}}%
\>[B]{}\F{Ren}\;\mathbin{:}\;\D{Cx}\;\blue{\rightarrow}\;\D{Cx}\;\blue{\rightarrow}\;\D{Set}{}\<[E]%
\\
\>[B]{}\F{Ren}\;\V{\Gamma}\;\V{\Delta}\;{}\<[14]%
\>[14]{}\mathrel{=}\;\mathkw{forall}\;\{\mskip1.5mu \V{\tau}\mskip1.5mu\}\;\blue{\rightarrow}\;\V{\tau}\;\D{\in}\;\V{\Gamma}\;\blue{\rightarrow}\;\V{\tau}\;\D{\in}\;\V{\Delta}{}\<[E]%
\\[\blanklineskip]%
\>[B]{}\F{Sub}\;\mathbin{:}\;\D{Cx}\;\blue{\rightarrow}\;\D{Cx}\;\blue{\rightarrow}\;\D{Set}{}\<[E]%
\\
\>[B]{}\F{Sub}\;\V{\Gamma}\;\V{\Delta}\;{}\<[14]%
\>[14]{}\mathrel{=}\;\mathkw{forall}\;\{\mskip1.5mu \V{\tau}\mskip1.5mu\}\;\blue{\rightarrow}\;\V{\tau}\;\D{\in}\;\V{\Gamma}\;\blue{\rightarrow}\;\V{\Delta}\;\D{\vdash}\;\V{\tau}{}\<[E]%
\ColumnHook
\end{hscode}\resethooks

The main issue when using DeBruijn indexes is the problematic (and boring...) index shift when
context grows at each abstraction.

\begin{hscode}\SaveRestoreHook
\column{B}{@{}>{\hspre}l<{\hspost}@{}}%
\column{18}{@{}>{\hspre}l<{\hspost}@{}}%
\column{E}{@{}>{\hspre}l<{\hspost}@{}}%
\>[B]{}\V{\char95 <><\char95 }\;\mathbin{:}\;\D{Cx}\;\blue{\rightarrow}\;\V{List}\;\D{Ty}\;\blue{\rightarrow}\;\D{Cx}{}\<[E]%
\\
\>[B]{}\V{xz}\;\V{<><}\;\V{<>}\;{}\<[18]%
\>[18]{}\mathrel{=}\;\V{xz}{}\<[E]%
\\
\>[B]{}\V{xz}\;\V{<><}\;(\V{x},\V{xs})\;{}\<[18]%
\>[18]{}\mathrel{=}\;\V{xz}\;\!\raisebox{ -0.09in}[0in][0in]{\red{\textsf{`}}\,}\;\V{x}\;\V{<><}\;\V{xs}{}\<[E]%
\\[\blanklineskip]%
\>[B]{}\mathkw{infixl}\;\V{4}\;\V{\char95 <><\char95 }{}\<[E]%
\ColumnHook
\end{hscode}\resethooks

We may then define the \emph{shiftable} simultaneous substitutions
from \ensuremath{\V{\Gamma}} to \ensuremath{\V{\Delta}}
as type-preserving mappings from the variables in any extension of \ensuremath{\V{\Gamma}} to
terms in the same extension of \ensuremath{\V{\Delta}}.
\begin{hscode}\SaveRestoreHook
\column{B}{@{}>{\hspre}l<{\hspost}@{}}%
\column{E}{@{}>{\hspre}l<{\hspost}@{}}%
\>[B]{}\F{Shub}\;\mathbin{:}\;\D{Cx}\;\blue{\rightarrow}\;\D{Cx}\;\blue{\rightarrow}\;\D{Set}{}\<[E]%
\\
\>[B]{}\F{Shub}\;\V{\Gamma}\;\V{\Delta}\;\mathrel{=}\;\mathkw{forall}\;\V{\Xi}\;\blue{\rightarrow}\;\F{Sub}\;(\V{\Gamma}\;\V{<><}\;\V{\Xi})\;(\V{\Delta}\;\V{<><}\;\V{\Xi}){}\<[E]%
\ColumnHook
\end{hscode}\resethooks

By the computational behaviour of \ensuremath{\V{<><}}, a \ensuremath{\F{Shub}\;\V{\Gamma}\;\V{\Delta}} can be used
as a \ensuremath{\F{Shub}\;(\V{\Gamma}\;\!\raisebox{ -0.09in}[0in][0in]{\red{\textsf{`}}\,}\;\V{\sigma})\;(\V{\Delta}\;\!\raisebox{ -0.09in}[0in][0in]{\red{\textsf{`}}\,}\;\V{\sigma})}, so we can push substitutions under
binders very easily.
\begin{hscode}\SaveRestoreHook
\column{B}{@{}>{\hspre}l<{\hspost}@{}}%
\column{22}{@{}>{\hspre}l<{\hspost}@{}}%
\column{E}{@{}>{\hspre}l<{\hspost}@{}}%
\>[B]{}\us{\F{/\!\!/}}\;\mathbin{:}\;\mathkw{forall}\;\{\mskip1.5mu \V{\Gamma}\;\V{\Delta}\mskip1.5mu\}\;(\V{\theta}\;\mathbin{:}\;\F{Shub}\;\V{\Gamma}\;\V{\Delta})\;\{\mskip1.5mu \V{\tau}\mskip1.5mu\}\;\blue{\rightarrow}\;\V{\Gamma}\;\D{\vdash}\;\V{\tau}\;\blue{\rightarrow}\;\V{\Delta}\;\D{\vdash}\;\V{\tau}{}\<[E]%
\\
\>[B]{}\V{\theta}\;\F{/\!\!/}\;\C{var}\;\V{i}\;{}\<[22]%
\>[22]{}\mathrel{=}\;\V{\theta}\;\V{<>}\;\V{i}{}\<[E]%
\\
\>[B]{}\V{\theta}\;\F{/\!\!/}\;\C{lam}\;\{\mskip1.5mu \V{\sigma}\mskip1.5mu\}\;\V{t}\;{}\<[22]%
\>[22]{}\mathrel{=}\;\C{lam}\;((\V{λ}\;\V{\Xi}\;\V{→}\;\V{\theta}\;(\V{\sigma},\V{\Xi}))\;\F{/\!\!/}\;\V{t}){}\<[E]%
\\
\>[B]{}\V{\theta}\;\F{/\!\!/}\;\C{app}\;\V{f}\;\V{s}\;{}\<[22]%
\>[22]{}\mathrel{=}\;\C{app}\;(\V{\theta}\;\F{/\!\!/}\;\V{f})\;(\V{\theta}\;\F{/\!\!/}\;\V{s}){}\<[E]%
\ColumnHook
\end{hscode}\resethooks

Of course, we shall need to construct some of these joyous shubstitutions.
Let us first show that any simultaneous renaming can be made shiftable by
iterative weakening.

\begin{hscode}\SaveRestoreHook
\column{B}{@{}>{\hspre}l<{\hspost}@{}}%
\column{8}{@{}>{\hspre}l<{\hspost}@{}}%
\column{16}{@{}>{\hspre}l<{\hspost}@{}}%
\column{18}{@{}>{\hspre}l<{\hspost}@{}}%
\column{E}{@{}>{\hspre}l<{\hspost}@{}}%
\>[B]{}\F{wkr}\;\mathbin{:}\;{}\<[8]%
\>[8]{}\mathkw{forall}\;\{\mskip1.5mu \V{\Gamma}\;\V{\Delta}\;\V{\sigma}\mskip1.5mu\}\;\blue{\rightarrow}\;\F{Ren}\;\V{\Gamma}\;\V{\Delta}\;\blue{\rightarrow}\;\F{Ren}\;(\V{\Gamma}\;\!\raisebox{ -0.09in}[0in][0in]{\red{\textsf{`}}\,}\;\V{\sigma})\;(\V{\Delta}\;\!\raisebox{ -0.09in}[0in][0in]{\red{\textsf{`}}\,}\;\V{\sigma}){}\<[E]%
\\
\>[B]{}\F{wkr}\;\V{r}\;\V{here}\;{}\<[16]%
\>[16]{}\mathrel{=}\;\V{here}{}\<[E]%
\\
\>[B]{}\F{wkr}\;\V{r}\;(\V{there}\;\V{i})\;{}\<[18]%
\>[18]{}\mathrel{=}\;\V{there}\;(\V{r}\;\V{i}){}\<[E]%
\\[\blanklineskip]%
\>[B]{}\F{ren}\;\mathbin{:}\;{}\<[8]%
\>[8]{}\mathkw{forall}\;\{\mskip1.5mu \V{\Gamma}\;\V{\Delta}\mskip1.5mu\}\;\blue{\rightarrow}\;\F{Ren}\;\V{\Gamma}\;\V{\Delta}\;\blue{\rightarrow}\;\F{Shub}\;\V{\Gamma}\;\V{\Delta}{}\<[E]%
\\
\>[B]{}\F{ren}\;\V{r}\;\V{<>}\;{}\<[18]%
\>[18]{}\mathrel{=}\;\V{λ}\;\{\mskip1.5mu \V{\tau}\mskip1.5mu\}\;\V{z}\;\V{→}\;\C{var}\;(\V{r}\;\V{z}){}\<[E]%
\\
\>[B]{}\F{ren}\;\V{r}\;(\anonymous ,\V{\Xi})\;{}\<[18]%
\>[18]{}\mathrel{=}\;\F{ren}\;(\F{wkr}\;\V{r})\;\V{\Xi}{}\<[E]%
\ColumnHook
\end{hscode}\resethooks

With renaming available, we can play the same game for substitutions.
\begin{hscode}\SaveRestoreHook
\column{B}{@{}>{\hspre}l<{\hspost}@{}}%
\column{8}{@{}>{\hspre}l<{\hspost}@{}}%
\column{16}{@{}>{\hspre}l<{\hspost}@{}}%
\column{18}{@{}>{\hspre}l<{\hspost}@{}}%
\column{E}{@{}>{\hspre}l<{\hspost}@{}}%
\>[B]{}\F{wks}\;\mathbin{:}\;{}\<[8]%
\>[8]{}\mathkw{forall}\;\{\mskip1.5mu \V{\Gamma}\;\V{\Delta}\;\V{\sigma}\mskip1.5mu\}\;\blue{\rightarrow}\;\F{Sub}\;\V{\Gamma}\;\V{\Delta}\;\blue{\rightarrow}\;\F{Sub}\;(\V{\Gamma}\;\!\raisebox{ -0.09in}[0in][0in]{\red{\textsf{`}}\,}\;\V{\sigma})\;(\V{\Delta}\;\!\raisebox{ -0.09in}[0in][0in]{\red{\textsf{`}}\,}\;\V{\sigma}){}\<[E]%
\\
\>[B]{}\F{wks}\;\V{s}\;\V{here}\;{}\<[16]%
\>[16]{}\mathrel{=}\;\C{var}\;\V{here}{}\<[E]%
\\
\>[B]{}\F{wks}\;\V{s}\;(\V{there}\;\V{i})\;{}\<[18]%
\>[18]{}\mathrel{=}\;\F{ren}\;\V{there}\;\F{/\!\!/}\;\V{s}\;\V{i}{}\<[E]%
\\[\blanklineskip]%
\>[B]{}\F{sub}\;\mathbin{:}\;{}\<[8]%
\>[8]{}\mathkw{forall}\;\{\mskip1.5mu \V{\Gamma}\;\V{\Delta}\mskip1.5mu\}\;\blue{\rightarrow}\;\F{Sub}\;\V{\Gamma}\;\V{\Delta}\;\blue{\rightarrow}\;\F{Shub}\;\V{\Gamma}\;\V{\Delta}{}\<[E]%
\\
\>[B]{}\F{sub}\;\V{s}\;\V{<>}\;{}\<[18]%
\>[18]{}\mathrel{=}\;\V{s}{}\<[E]%
\\
\>[B]{}\F{sub}\;\V{s}\;(\anonymous ,\V{\Xi})\;{}\<[18]%
\>[18]{}\mathrel{=}\;\F{sub}\;(\F{wks}\;\V{s})\;\V{\Xi}{}\<[E]%
\ColumnHook
\end{hscode}\resethooks


\end{document}
